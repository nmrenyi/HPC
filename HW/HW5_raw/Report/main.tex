%%%%%%%%%%%%%%%%%%%%%%%%%%%%%%%%%%%%%%%%%

% Lachaise Assignment
% LaTeX Template
% Version 1.0 (26/6/2018)
%
% This template originates from:
% http://www.LaTeXTemplates.com
%
% Authors:
% Marion Lachaise & François Févotte
% Vel (vel@LaTeXTemplates.com)
%
% License:
% CC BY-NC-SA 3.0 (http://creativecommons.org/licenses/by-nc-sa/3.0/)
% 
%%%%%%%%%%%%%%%%%%%%%%%%%%%%%%%%%%%%%%%%%

%----------------------------------------------------------------------------------------
%	PACKAGES AND OTHER DOCUMENT CONFIGURATIONS
%----------------------------------------------------------------------------------------

\documentclass[UTF8]{article}

\input{structure.tex} % Include the file specifying the document structure and custom commands

%----------------------------------------------------------------------------------------
%	ASSIGNMENT INFORMATION
%----------------------------------------------------------------------------------------

\title{Introduction to HPC \\ HW5 Report} % Title of the assignment

\author{姓名:任一  \\学号:2018011423\\ \texttt{ry18@mails.tsinghua.edu.cn}} % Author name and email address

\date{\today} % University, school and/or department name(s) and a date

%----------------------------------------------------------------------------------------
\lstset{
    % backgroundcolor=\color{red!50!green!50!blue!50},%代码块背景色为浅灰色
    rulesepcolor= \color{gray}, %代码块边框颜色
    breaklines=true,  %代码过长则换行
    numbers=left, %行号在左侧显示
    numberstyle= \small,%行号字体
    keywordstyle= \color{blue},%关键字颜色
    commentstyle=\color{gray}, %注释颜色
    frame=shadowbox%用方框框住代码块
}

\begin{document}

\maketitle % Print the title
\begin{center}
    \begin{tabular}{l  r}
    \hline
        \multicolumn{2}{c}{实验环境} \\ \hline
        操作系统: & Windows10家庭版 18362.72 Windows Subsystem for Linux \\ \hline% Date the experiment was performed
        gcc版本: & gcc version 7.5.0 \\ \hline% Partner names
    \end{tabular}
\end{center}
\newpage

\section{Exercise5.4}
各规约操作符与其初始化的变量值如下表:
\begin{table}[h]
    \label{tab:my-table}
    \centering
    \caption{规约运算符与其对应的变量初始值表}
    \begin{tabular}{|l|l|}
    \hline
    Operator           & Initial Value \\ \hline
    \&\&               & $1$             \\ \hline
    ||                 & $0$             \\ \hline
    \&                 & $111...111_2$     \\ \hline
    |                  & $0$             \\ \hline
    \textasciicircum{} & $0$             \\ \hline
    \end{tabular}
    \end{table}

\section{Exercise5.5}
\subsection{1}
串行相加时,当完成最后一次相加后,寄存器中的数值是$1.008e+03$, 
当这个数字被存储到内存中时,该数值被四舍五入为3位十进制有效数字,
即$sum = 1.01e+03$. 因此输出的值为1010.0.

\subsection{2}
使用2线程并行相加时,
0号线程负责前两个数的相加,局部和为$4.00e+00$, 1号线程负责后两个数
的相加,局部和为$1.00e+03$(此处发生了四舍五入). 
这两个线程局部和相加,得到的结果为$1.00+03$(此处发生了四舍五入).
因此输出的值为$1000.0$.

\section{PA2}
\subsection{}
私有的变量有i, j, count. 共享的变量有a, n, temp.
\subsection{}
没有循环依赖。每个线程之间彼此都使用私有的i, j, count,
对共享的temp数组的写操作也是彼此不重叠的,对a和n只存在读操作,
因此不存在依赖和冲突。
\subsection{}
可以并行化memcpy的调用。但是不一定能提高效率,
因为memcpy自身效率已经很高,开辟新线程还会带来额外的时间开销,因此不一定能够提高效率。

\begin{figure}[h]
    \label{Ratio}
    \centering
        \includegraphics[width=0.8\textwidth]{Ratio.png}
        \caption{我的算法与课程提供的策略文件博弈胜率折线图}
    \end{figure}


\bibliographystyle{plain}
\bibliography{ref} %这里的这个ref就是对文件ref.bib的引用

\end{document}
